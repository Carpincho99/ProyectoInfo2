Para conseguir esta secuencia se define, primeramente, un arreglo de tipo uint8 t (unsigned char) constante donde serán almacenados los pines y el puerto correspondiente a cada entrada del driver
Estos serán pasados, junto al número de pasos a ejecutar, por referencia y copia respectivamente, a la función moveAxisRelative. Esta define un contador para indicar el paso actual, por lo que varía entre 0 y 3. Por último, se hace uso de un arreglo de funciones, doStepHorario, para llamar a la función del paso que corresponda. Función que es la encargada de apagar y energizar las bobinas del motor por medio del registro de 8 bits correspondiente al puerto, y con el uso de los operadores a nivel de bit y sentencias correspondientes para apagar y encender respectivamente.
\par
Para la apertura y cierre del servomotor, se hizo uso de los timers del propio Atmega328 para generar una señal de pulso modulado (pwm). De este modo, 
haciendo variar el duty cycle de la señal, conseguimos que el servo motor se posición según lo deseado. 

