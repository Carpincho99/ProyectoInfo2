Dentro del código se incluyen todos los archivos de cabecera dentro de un único llamado allInc.h. 
Este último contiene, entre otras cosas, las funciones SetInOut, pwm init y las relacionadas a la Uart.
\par
Con las funciones SetInOut se inicializan los pines del arduino, con UART init 
se configura la velocidad para la uart y la comunicación serial y con pwm init 
se inicializa el pwm y se setea el valor mínimo $(0)$ y máximo $(180)$ del 
servomotor para la amplitud de la apertura de la pinza.
En el bucle infinito, la función UART gets lee el puerto de serie hasta que 
haya un salto de línea, almacenando el dato en la variable line para luego 
traducirlo con la función parse en comandos de movimiento, devolviendo error u 
ok según corresponda.

