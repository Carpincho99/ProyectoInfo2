Nos permite ingresar una cadena de caracteres en donde discrimina los espacios en blanco. Posteriormente mediante la función strToUpper nos permite hacer la conversión a una letra Mayúscula. Esto nos permite mayor tolerancia al momento de ingresar en qué eje se ha de desplazar el brazo, tomando en cuenta que si se ingresa una letra Minúscula hará la conversión correspondiente y permitirá realizar la acción de los motores paso a paso. En dado caso en que se ingrese un carácter S ejecutará la acción de la función execLine que hará accionar el servo.
\par
Dado el caso en el que se ingrese x,y,z por teclado y también se ingrese un valor por ejemplo 100, la letra indicará hacia que eje debe hacer el movimiento el brazo y el valor indicará la cantidad de paso(step) que debe hacer en dicha dirección.
\par
El programa contempla el sentido en que debe desplazarse el brazo, dado que puede realizar el desplazamiento en ambos sentidos.
